\usepackage{fancyhdr}
\usepackage{geometry}
\usepackage[center,small,bf]{caption}

\pagestyle{fancy}

\fancyhf{} %borra todos los ajustes
\fancyhead[R]{\bfseries\thepage} %coloca el numero de pagina a la derecha en cada capitulo en bold
\fancyhead[L]{\bfseries\leftmark} %coloca el titulo del capitulo a la izquierda en bold

% Cambia la estructura de una pagina en blanco
\fancypagestyle{plain}{
\fancyhead{}
\renewcommand{\headrulewidth}{0pt}
}

% Cambia el ancho del encabezado
\setlength{\headwidth}{16.5cm}

% Cambia el espacio para el encabezado
\setlength{\headheight}{20pt}

% Cambia el margen de los pies de figura
\setlength{\captionmargin}{20pt}

% Borra la palabra Capítulo del \chaptermark:
\renewcommand{\chaptermark}[1]{\markboth{\textbf{\thechapter. #1}}{}} % Formato para el capítulo: N. Nombre

% Define comando para colocar una pagina en blanco antes de iniciar cada capítulo
\newcommand{\paginaenblanco}{\clearpage
\thispagestyle{empty}
\vfill
\newpage{\tiny {.}}
\vfill
\addtocounter{page}{-1}}
%\newcommand{\clearemptydoublepage}{\newpage{\pagestyle{empty}
%\cleardoublepage}}
%\newcommand{\HRule}{\rule{\linewidth}{0.5mm}}

