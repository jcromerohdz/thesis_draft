\chapter{Caso de estudio}

Describir lo que se va a analizar
universidades. Describir cada una de las escuelas
lugar
no de estudiantes
Estadisticas 
Cursos comunes 
Colocar prograsmas de estudio de cada carrera y escuela
Por que se propone el caso de estudio?


\section{Descripci\'on del Caso de Estudio}

Los estudiantes de ingenier\'{i}a utilizan diferentes formas para aprender, usan diferentes recursos dependiento de sus propio estilo de aprendizaje los cuales muestran diferentes preferencias para ello. Los objetos de aprendizaje podr\'{i}an ser m\'{a}s efectivos  cuando se muestran recursos a los usuarios de los sistemas de aprendizaje de una manera personalizada. Este caso de estudio esta enfocado en describir las preferencias de los objetos de aprendizaje del test de los estilos de aprendizaje VARK de Fleming  de estudiantes de nivel licenciatura de ingenier\'{i}a. 

Aplicamos una encuesta de 1200 cuestionarios  a estudiantes de ingenier\'ia, pero s\'{o}lo 1042 de estos cuestionarios fueron completados satisfactoriamente. La encuesta recupera informaci\'{o}n sobre los estilos de aprendizaje  utilizando el test standar VARK y un cuestionario sobre las preferencias de los objetos de aprendizaje utilizando la escala que va de menor a mayor preferencia, la cual se indica por medio de una barra de desplazamiento.
 
Estos estudiantes fueron de dos universidades p\'{u}blicas  en Baja California, M\'{e}xico: Instituto Tecnol\'{o}gico de Tijuana (ITT) con los estudiantes de Ingenier\'{i}a en Sistemas Computacionales e Ingenier\'{i}a en Tecnolog\'{i}as de la Informaci\'{o}n y Comunicaciones; y la Univeridad Aut\'{o}noma de Baja California (UABC) con los estudiantes de Ingenier\'{i}a en Computaci\'{o}n.
   
\section{Aplicaci\'on de los Cuestionarios VARK, IM y Preferencias del Usuario} 

Para obtener la informaci\'{o}n acerca de los estilos de aprendizaje  de los usuarios. utilizamos la versi\'{o}n m\'{a}s reciente del cuestionario VARK de Fleming, el cuestionario de Gardner para las Inteligencias M\'ultiples y el cuestionario de Preferencias de estudios de los estudiantes. Con estos cuestionarios, conseguimos informaci\'{o}n acerca de sus estilos de aprendizaje, inteligencias m\'ultiples y preferencias de estudio. Con el cuestionario de preferencias se obtuvo informaci\'on de como realizan una tarea espec\'{i}fica  para aprender o para resolver problemas.

Para realizar los cuesionarios de Preferencias, agrupamos las preguntas y posibles respuestas de acuerdo a la recomendaci\'{o}n de unos expertos; estos expertos fueron docentes que ense\~{n}aban computaci\'{o}n en el ITT y en la UABC, espec\'{i}ficamente la materia de "programaci\'{o}n orientada a objetos". Cada pregunta sobre las preferencias en el cuestionario tienen una barra donde el estudiante la puede desplazar de no-preferencia (lado iquierdo de la barra) hasta una preferencia alta (lado derecho de la barra) para cada una de las preguntas a sus preferencias y de como ellos trabajan en clases. Los valores obtenidos del cuestionario estan entre 0 y 1, donde 0 significa que el estudiante no tiene ningna preferencia y 1 significa que tiene la m\'{a}xima preferencia. 

Se aplic\'{o} el cuestionario para obtenter las preferencias del usuario. Se hicieron preguntas acerca de las ayudas sobre preferencias visuales de los estudiantes, tales como: fotos, diagramas, videos, televisi\'{o}n , videos en internet para explicar o para aprender entre ellos.

Para las preferencias auditivas, se hicieron preguntas como: si utilizan el celular para intercambiar informaci\'{o}n, si utilizan la radio para escucharla o para escuchar  m\'{u}sica para cuando estan estudiando, si alguno de sus compa\~{n}eros les explican algo de lo que ellos no entiendan o que quieran saber, y tambi\'{en} sus preferencias por ex\'{a}menes orales.

Para las preferencias lecto/escritura, se pregunt\'{o} lo siguiente: si leen el peri\'{o}dico, revistas, libros, etc., para cuando quieren buscar cierta informaci\'{o}n; si utiliza manuales, o si prefieren ex\'{a}menes te\'{o}ricos, si les gusta escribir y mandar mensajes por el celular o el twitter y si le gustan las plataformas virtuales.

Para las preferencias kinest\'{e}sicas, se hicieron preguntas como: si les gustan los ex\'{a}menes pr\'{a}cticos, relizar pr\'{a}cticas de laboratorio, preguntarle a los amigos o a un experto acerca de un tema, tambi\'{e}n si les gustan las clases presenciales o semipresenciales y si les gusta navegar cuando est\'{a}n ante un nuevo software mientras aprenden acerca de \'{e}l.   
   
   
\subsection{Dise\~{n}o de Cuesionarios VARK, IM y Preferencias del Usuario} 

Para obtener los datos de los estudiantes, se desarrollo una plataforma web, en la cual ellos pod\'ian llenar los cuestionarios.
   
\subsubsection{Cuestionario VARK}    

La Figura \ref{figure:VARKPreguntas} muestra una pantalla de como qued\'o el cuestionario VARK y la captura de datos del estudiante en la web.

\begin{figure}[!h]
\begin{center}
\includegraphics[width=4in]{imagenes/VARKPreguntas.jpg}
\caption{Pantalla de captura del cuestionario VARK y los datos del estudiante}
\label{figure:VARKPreguntas}
\end{center}
\end{figure}

\newpage
\subsubsection{Cuestionario IM} 

La Figura \ref{figure:IMPreguntas} muestra una pantalla de como qued\'o el cuestionario de IM en la web.

\begin{figure}[!h]
\begin{center}
\includegraphics[width=4in]{imagenes/IMPreguntas.jpg}
\caption{Pantalla de captura del cuestionario de IM}
\label{figure:IMPreguntas}
\end{center}
\end{figure}

La Figura \ref{figure:PreferenciasPreguntas} muestra una pantalla de como qued\'o el cuestionario de las preferencias del estudiante en la web.

\begin{figure}[!h]
\begin{center}
\includegraphics[width=4in]{imagenes/PreferenciasPreguntas.jpg}
\caption{Pantalla de captura de las preferencias de estudio}
\label{figure:PreferenciasPreguntas}
\end{center}
\end{figure}

\newpage
Los Cuestionarios completos pueden localizarse en el Apendice A.    


